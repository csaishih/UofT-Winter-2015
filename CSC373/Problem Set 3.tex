%%  Problem Set 2 for CSC373H1, Winter 2015,
%%  at the University of Toronto.
%%
%%  Copyright (c) 2015 Francois Pitt <fpitt@cs.utoronto.ca>
%%  last updated at 09:30 (EST) on Tue  6 Jan 2015
%%
%%  CC BY-SA 4.0
%%  This work (the current file named 'PS1handout.tex') is licensed under
%%  the Creative Commons Attribution-ShareAlike 4.0 International License.
%%  To view a copy of this license, visit
%%      http://creativecommons.org/licenses/by-sa/4.0/
%%  or send a letter to: Creative Commons, 444 Castro Street, Suite 900,
%%  Mountain View, California, 94041, USA.
%%  This is a human-readable summary of (and not a substitute for) the
%%  license.
%%  You are free to:
%%      Share -- copy and redistribute the material in any medium or format
%%      Adapt -- remix, transform, and build upon the material for any
%%          purpose, even commercially.
%%      The licensor cannot revoke these freedoms as long as you follow the
%%          license terms.
%%  Under the following terms:
%%      Attribution -- You must give appropriate credit, provide a link to
%%          the license, and indicate if changes were made. You may do so in
%%          any reasonable manner, but not in any way that suggests the
%%          licensor endorses you or your use.
%%      ShareAlike -- If you remix, transform, or build upon the material,
%%          you must distribute your contributions under the same license as
%%          the original.
%%      No additional restrictions -- You may not apply legal terms or
%%          technological measures that legally restrict others from doing
%%          anything the license permits.
%%  Notices:
%%      You do not have to comply with the license for elements of the
%%      material in the public domain or where your use is permitted by an
%%      applicable exception or limitation.
%%      No warranties are given. The license may not give you all of the
%%      permissions necessary for your intended use. For example, other
%%      rights such as publicity, privacy, or moral rights may limit how you
%%      use the material.

\RequirePackage[l2tabu,orthodox]{nag}  % warn about common LaTeX pitfalls
\RequirePackage[ascii]{inputenc}  % input is 7-bit ASCII
\RequirePackage{fixltx2e}  % fix LaTeX2e kernel bugs

\documentclass[11pt,twoside]{article}
\usepackage{calc}
\usepackage{enumitem}
\usepackage{fancyhdr}
\usepackage[hmargin=1in,vmargin=1in]{geometry}
\usepackage{lastpage}
\usepackage{relsize}
\usepackage[normalem]{ulem}
\usepackage{url}
\usepackage{amsmath}
\usepackage{amssymb}

% Set up fonts.
\usepackage[T1]{fontenc}  % use true 8-bit fonts
\usepackage{slantsc}  % allow slanted small-caps
\usepackage{microtype}  % perform various font optimizations
% Use Palatino-based monospace instead of kpfonts' default.
\usepackage{newpxtext}
\ttfamily
\DeclareFontShape{T1}{\ttdefault}{m}{scsl}{<->ssub*\ttdefault/m/sc}{}
\DeclareFontShape{T1}{\ttdefault}{b}{scsl}{<->ssub*\ttdefault/b/sc}{}
% "Kepler" fonts.
\usepackage[nott,notextcomp]{kpfonts}
% Use curvier Latin Modern brackets instead of kpfonts' glyphs.
\DeclareSymbolFont{lmsymb}     {OMS}{lmsy}{m}{n}
\DeclareSymbolFont{lmlargesymb}{OMX}{lmex}{m}{n}
\DeclareMathDelimiter{\rbrace}{\mathclose}{lmsymb}{"67}{lmlargesymb}{"09}
\DeclareMathDelimiter{\lbrace}{\mathopen}{lmsymb}{"66}{lmlargesymb}{"08}

%% For drawing diagrams.
%\usepackage{tikz}
%\usetikzlibrary{positioning}

% Page layout: stretch text to fill up page.
\addtolength\footskip{.25\headheight}
\flushbottom

% Common list settings.
\setlist{topsep=\parsep,itemsep=0pt}  % no extra vertical space around lists

% Common macros.

% Headings.
\pagestyle{fancy}
\let\headrule\empty
\let\footrule\empty
\lhead{\scshape CSC\,373\,H1}
\chead{\large\scshape Problem Set \#\,2}
\rhead{\scshape Winter 2015}
\lfoot{\scshape Shihan Ai, 999700649, g3aishih}
\cfoot{\scshape}
\rfoot{\scshape page \thepage\space of \pageref{LastPage}}

\begin{document}
{\bf I used this website to help me in this problem set:}\\ http://stackoverflow.com/questions/2633848/dynamic-programming-coin-change-decision-problem
\begin{enumerate}[leftmargin=0pt,label=(\alph*)]

\item {\bf Recursive structure}\\
The solution for N depends on solutions for $N-N=0$, $N-3$, $N-10$, and $N-25$ (if they exist, i.e if $N=3$ then there is no solution for $N-10$ or $N-25$ since $N\geq0$)\\
If we run our algorithm on an input {\bf N} that {\bf should} return {\bf True}, that means that {\bf N} is the {\bf exact} sum of a certain number of 3's, a certain number of 10's and a certain number of 25's (a, b, c are integers greater than or equal to 0) {\bf N = 3a + 10b + 25c}\\
That also means that either 3, 10 or 25 can be subtracted from {\bf N}. If we subtract either 3, 10, or 25, we know that {\bf at least one} of the differences {\bf D} where {\bf D} can be \{$0$, $N-3$, $N-10$, $N-25$ such that {\bf D} $\geq0$\} must be an {\bf exact} sum of a certain number of 3's, a certain number of 10's and a certain number of 25's in order for the algorithm to return {\bf True} on {\bf N} (i.e running the algorithm on {\bf D} should return {\bf True}). This is where we can check if the output of running our algorithm on {\bf D} will return {\bf True}. If it does return {\bf True} then you should be able to buy exactly {\bf N} nuggets at this restaurant, if not, then you should {\bf not} be able to buy exactly {\bf N} nuggets at this restaurant. This is because if you currently have a sum that is not exactly composed of a certain number of 3's, a certain number of 10's and a certain number of 25's, then adding 3, 10, or 25 will not change the fact that it's not an exact sum of 3's, 10's and 25's.

\item We define an array OPT of size $N+1$\\
$OPT[i],\ i\ =\ 0,1,2,...,N$\\
where $OPT[i]$ contains a boolean value indicating whether or not you can purchase exactly $i$ Tofu Nuggets at this restaurant\\
In other words, $OPT[i]$ stores the output of running the algorithm on input $i$

\item {\bf Base case:} $OPT[0]$ = True\\
I can purchase exactly $0$ nuggets by purchasing exactly $0$ boxes of 3, 10, or 25 nuggets\\
{\bf General case:} {\bf for} $i = 1\ ...\ N$\\
$OPT[i]$ = {\bf False}\hspace{2mm} {\bf unless} \hspace{2mm}$i\geq j$ and $OPT[i-j]$ = True \hspace{2mm} for j = any value in the set \{3, 10, 25\}\\ If this is the case {\bf then} $OPT[i]$ = {\bf True}\\
We know from the structure described in part {\bf (a)} that the algorithm must return {\bf True} for a number {\bf D} where {\bf D} can be \{$0$, $N-3$, $N-10$, $N-25$ such that {\bf D} $\geq0$\} in order for the algorithm to return {\bf True} for {\bf N}. This is where the intuition for my recurrence relation came from.

\item Dynamic(N)
\begin{enumerate}[leftmargin=10pt,label=\arabic*]
\item OPT[0] = True
\item \hspace{2mm}{\bf for} i = 1 .. N
\item \hspace{2mm}\hspace{5mm}OPT[i] = False
\item \hspace{2mm}\hspace{5mm}{\bf for} j = (3, 10, 25)
\item \hspace{2mm}\hspace{5mm}\hspace{5mm}{\bf if} i $\geq$ j {\bf and} OPT[i $-$ j] \hspace{10mm}//If i < j, then the first condition will be false and should
\item \hspace{2mm}\hspace{5mm}\hspace{5mm}\hspace{5mm}OPT[i] = True \hspace{10mm}//exit the if statement, in other words, we shouldn't get a negative
\item \hspace{2mm}{\bf return} OPT[N] \hspace{20mm}//index for OPT for the second condition check
\end{enumerate}
RunTime: $\Theta$(n)\\
Yes, it runs in polynomial time since:\\
{\bf for} loop at line 2 runs N times\\
Nested {\bf for} loop at line 4 runs 3 times\\
RunTime: $\Theta$(3n) = $\Theta$(n)\\


\end{enumerate}
\end{document}