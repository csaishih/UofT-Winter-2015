%%  Problem Set 1 for CSC373H1, Winter 2015,
%%  at the University of Toronto.
%%
%%  Copyright (c) 2015 Francois Pitt <fpitt@cs.utoronto.ca>
%%  last updated at 09:30 (EST) on Tue  6 Jan 2015
%%
%%  CC BY-SA 4.0
%%  This work (the current file named 'PS1handout.tex') is licensed under
%%  the Creative Commons Attribution-ShareAlike 4.0 International License.
%%  To view a copy of this license, visit
%%      http://creativecommons.org/licenses/by-sa/4.0/
%%  or send a letter to: Creative Commons, 444 Castro Street, Suite 900,
%%  Mountain View, California, 94041, USA.
%%  This is a human-readable summary of (and not a substitute for) the
%%  license.
%%  You are free to:
%%      Share -- copy and redistribute the material in any medium or format
%%      Adapt -- remix, transform, and build upon the material for any
%%          purpose, even commercially.
%%      The licensor cannot revoke these freedoms as long as you follow the
%%          license terms.
%%  Under the following terms:
%%      Attribution -- You must give appropriate credit, provide a link to
%%          the license, and indicate if changes were made. You may do so in
%%          any reasonable manner, but not in any way that suggests the
%%          licensor endorses you or your use.
%%      ShareAlike -- If you remix, transform, or build upon the material,
%%          you must distribute your contributions under the same license as
%%          the original.
%%      No additional restrictions -- You may not apply legal terms or
%%          technological measures that legally restrict others from doing
%%          anything the license permits.
%%  Notices:
%%      You do not have to comply with the license for elements of the
%%      material in the public domain or where your use is permitted by an
%%      applicable exception or limitation.
%%      No warranties are given. The license may not give you all of the
%%      permissions necessary for your intended use. For example, other
%%      rights such as publicity, privacy, or moral rights may limit how you
%%      use the material.

\RequirePackage[l2tabu,orthodox]{nag}  % warn about common LaTeX pitfalls
\RequirePackage[ascii]{inputenc}  % input is 7-bit ASCII
\RequirePackage{fixltx2e}  % fix LaTeX2e kernel bugs

\documentclass[11pt,twoside]{article}
\usepackage{calc}
\usepackage{enumitem}
\usepackage{fancyhdr}
\usepackage[hmargin=1in,vmargin=1in]{geometry}
\usepackage{lastpage}
\usepackage{relsize}
\usepackage[normalem]{ulem}
\usepackage{url}
\usepackage{amsmath}
\usepackage{amssymb}

% Set up fonts.
\usepackage[T1]{fontenc}  % use true 8-bit fonts
\usepackage{slantsc}  % allow slanted small-caps
\usepackage{microtype}  % perform various font optimizations
% Use Palatino-based monospace instead of kpfonts' default.
\usepackage{newpxtext}
\ttfamily
\DeclareFontShape{T1}{\ttdefault}{m}{scsl}{<->ssub*\ttdefault/m/sc}{}
\DeclareFontShape{T1}{\ttdefault}{b}{scsl}{<->ssub*\ttdefault/b/sc}{}
% "Kepler" fonts.
\usepackage[nott,notextcomp]{kpfonts}
% Use curvier Latin Modern brackets instead of kpfonts' glyphs.
\DeclareSymbolFont{lmsymb}     {OMS}{lmsy}{m}{n}
\DeclareSymbolFont{lmlargesymb}{OMX}{lmex}{m}{n}
\DeclareMathDelimiter{\rbrace}{\mathclose}{lmsymb}{"67}{lmlargesymb}{"09}
\DeclareMathDelimiter{\lbrace}{\mathopen}{lmsymb}{"66}{lmlargesymb}{"08}

%% For drawing diagrams.
%\usepackage{tikz}
%\usetikzlibrary{positioning}

% Page layout: stretch text to fill up page.
\addtolength\footskip{.25\headheight}
\flushbottom

% Common list settings.
\setlist{topsep=\parsep,itemsep=0pt}  % no extra vertical space around lists

% Common macros.

% Headings.
\pagestyle{fancy}
\let\headrule\empty
\let\footrule\empty
\lhead{\scshape CSC\,373\,H1\ Shihan\, Ai\, 999700649}
\chead{\large\scshape Problem Set \#\,1}
\rhead{\scshape Winter 2015}
\lfoot{\scshape Shihan Ai, 999700649, g3aishih}
\cfoot{\scshape}
\rfoot{\scshape page \thepage\space of \pageref{LastPage}}


\begin{document}
\begin{enumerate}[leftmargin=0pt,label=(\alph*)]
\item ALGORITHM(L)\hspace{26mm}{\scshape //L is the list of processing times}
    \begin{enumerate}[leftmargin=10pt,label=\arabic*]
    \item\hspace{2mm} n = |L| \hspace{35mm}{\scshape //get length of list input L}
    \item\hspace{2mm} R = []\hspace{38mm}{\scshape //output}
    \item\hspace{2mm} Lcopy = deepcopy(L)
    \item\hspace{2mm} sort(Lcopy)\hspace{28mm}{\scshape //sort in ascending order}
    \item\hspace{2mm} {\bf for}\ i\ =\ 0\ {\bf to}\ n\ $-$\ 1 \hspace{18mm}{\scshape //for each processing time {\bf P} in Lcopy, find the index of}
    \item\hspace{2mm} \hspace{5mm} P\ =\ Lcopy[i]\hspace{20mm}{\scshape //the first occurrence of {\bf P} in L}
    \item\hspace{2mm} \hspace{5mm} {\bf for}\ j\ =\ 0\ {\bf to}\ n\ $-$\ 1
    \item\hspace{2mm} \hspace{5mm} \hspace{5mm} {\bf if} L[j]\ =\ P
    \item\hspace{2mm} \hspace{5mm} \hspace{5mm} \hspace{5mm} L[j]\ =\ $-$1 \hspace{13mm}{\scshape //found an occurrence, set the occurrence to $-$1 so we don't}
    \item\hspace{2mm} \hspace{5mm} \hspace{5mm} \hspace{5mm} R.append(j+1) \hspace{3mm}{\scshape //return this index again in case there are duplicates of}
    \item\hspace{2mm} \hspace{5mm} \hspace{5mm} \hspace{5mm} {\bf break} \hspace{18mm}{\scshape //this occurrence in L}
    \item\hspace{2mm}{\bf return} R \hspace{32mm}{\scshape //no elements can be $-$1 in L originally since $t_{1},...,t_{n}\ \in\ \mathbb{N}$}

    \end{enumerate}
\item A partial solution for my algorithm is a list {\bf K} that contains the indices of processing time(s) sorted such that the average completion time of executing all process(es) in the order specified by {\bf K} is minimized.\\
{\bf Partial solutions:} (Let $R_{n}$ be the partial solution constructed by the algorithm after iteration $n$) \\
n = 0 \hspace{5mm} $R_{0}$ = []\\
n = 1 \hspace{5mm} $R_{1}$ = [1]\\
n = 2 \hspace{5mm} $R_{2}$ = [1, 4]\\
n = 3 \hspace{5mm} $R_{3}$ = [1, 4, 3]\\
n = 4 \hspace{5mm} $R_{4}$ = [1, 4, 3, 2]\\

\item Let $R_{0},\ R_{1},\ ...,\ R_{n}$ be the {\bf R} list at the end of each iteration of the loop on line 5. For any optimum solution OPT, we say OPT {\bf extends} $R_{i}$ if
    \begin{itemize}
    \item for $i>0,\ i\ \in\ \mathbb{Z},\ OPT[k]\ =\ R_{i}[k]$ for $k=0,...,m$ where $m=|R_{i}|\ -\ 1$
    \item for $i\ =\ 0,\ R_{i}\ =\ []$
    \end{itemize}
A partial solution, $R_{i}$ is said to be promising iff $\ \exists OPT$ that extends $R_{i}$

\item {\bf Proof:} $R_{i}$ is promising, for all $i$ where $i$ = $0,...,|L|-1$ by induction on $i$ (\# of iterations)

\item Lcopy contains $t_{1},...,t_{n}$ in a sorted ascending order. Let $T_{1},...,T_{n}$ be the elements of Lcopy.\\
In other words,\\
$T_{1}$ = some element in \{$t_{1},...,t_{n}$\}\\
$T_{2}$ = some element in (\{$t_{1},...,t_{n}$\} $-$ \{$T_{1}$\})\\
$T_{3}$ = some element in (\{$t_{1},...,t_{n}$\} $-$ \{$T_{1}, T_{2}$\})\\
$T_{n}$ = some element in (\{$t_{1},...,t_{n}$\} $-$ \{$T_{1},...,T_{n-1}$\})\\\\
Let $K_{1}$ be the index of $T_{1}$ in L, $K_{2}$ be the index of $T_{2}$ in L, ..., $K_{n}$ be the index of $T_{n}$ in L. The values of $K_{1},...,K_{n}$ are unique, for example, if some $T_{i}$ = $T_{j}$ then there will be at least 2 occurrences of the value {\bf G}, (G = $T_{i}\ =\ T_{j}$) in L. $K_{i}$ will be the index of one occurrence of {\bf G} in L and $K_{j}$ will be the index of another occurrence of {\bf G} in L\\\\
For each iteration, there is only one case:
    \begin{itemize}
    \item $R_{i+1}\ =\ R_{i}$ append $K_{i+1}$
    \end{itemize}
\end{enumerate}

\end{document}
